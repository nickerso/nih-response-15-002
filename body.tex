% -*- TeX-master: "main"; fill-column: 72 -*-
% ----------------------------------------------------------------------

% Miscellaneous macros used below.

\newcommand{\divider}[1]{\vspace{0.25em}\textbf{\large #1}~\xrfill[0pt]{0.5pt}[gray]}
\newcommand{\subdivider}[1]{\vspace{0.25em}\textbf{\textsl{#1}}}
\newcommand{\eg}{e.g.,\xspace}
\newcommand{\idorg}{\href{http://identifiers.org}{\texttt{Identifiers.org}}\xspace}

% Body.

Dear NIH,

We are the coordinators of COMBINE (Computational Modeling in Biology Network; \url{http://co.mbine.org}), an initiative to better organize the development of many popular standards and formats for computational modeling in biology.  We are grateful for the opportunity provided by NIH to respond to the \emph{Request for Information (RFI) Making Data Usable -- A Framework for Community-Based Data and Metadata Standards Efforts for NIH-relevant Research}.  We would like to summarize COMBINE's efforts, and address some of the questions posed in the RFI from the perspective of our experiences developing community-based standards in biology.


\divider{The history and goals of COMBINE}

The availability of appropriate data formats and process descriptions is an essential enabler for reproducible science.  Researchers must build on each other's work to develop a deeper understanding of biological phenomena, but that is impeded if they do not use common languages to describe their work.  This has led to the creation of many formats and minimum information guidelines to facilitate the exchange of data and models.  However, the existence of uncoordinated standards risks creating silos that induce new interoperability problems.

At one extreme, one can hold the position that it does not matter how many standards are proposed; natural selection will eventually lead the best to be adopted.  At the other extreme, one can impose standards choices in a top-down fashion.  The best long-term approach surely lies between these extremes.  The ineffectiveness of attempting to impose data standards in a top-down fashion in biology has been recognized~\cite{quackenbush_2006b}.  Our experiences likewise support the view that the most useful standards are those developed bottom-up by stakeholders who have a vested interest in making the standards truly work.  What is often missing, however, is better coordination and communication between groups, and better resources to streamline the standards development process.

% Sometimes simply learning of the \emph{existence} of a standardization effort can help prevent needless redundancy in an area.

COMBINE was formed in 2009 by the groups involved in developing file formats and other standards in systems biology, including SBML~\cite{hucka_2003, hucka_2010}, SBGN~\cite{le2009systems}, BioPAX~\cite{demir2010}, CellML~\cite{cuellar2003overview, hedley_2001b}, SED-ML~\cite{sedml2011}, SBOL~\cite{galdzicki2014}, NeuroML~\cite{gleeson_2010}, and others.  COMBINE aims to act as a coordinator, facilitator, and resource for different standardization efforts whose domains of use cover related areas of the computational biology space.  Our hope is that it can help the federated projects develop standards that are more interoperable and with less overlap than if the efforts proceeded separately.  We believe exchange and discussion between interested groups is an essential ingredient that not only helps decrease the creation of unnecessary standards due to lack of knowledge about what is available, but also improves the quality of---and interoperability between---related standards.  COMBINE does not dictate what individual standardization efforts should do, however; actions are entirely up to the leaders and members of the communities involved in the individual efforts.  COMBINE \emph{does} offer examples of what has worked in terms of community organization approaches, as well as some common infrastructure for such things as cataloguing standards specifications, and common meetings fostering greater interaction between the relevant communities.


\divider{Groups involved in COMBINE today}

COMBINE today includes the efforts listed in the following table.  All are open community efforts, with freely available specifications, open community participation, etc.  They cover a range of topics: raw data standards, model format standards, graphical notation standards, ontologies, and minimum information guidelines.

\newcommand{\URL}[1]{\textls[-25]{\url{http://co.mbine.org/standards/#1}}}
\newcolumntype{P}[1]{>{\raggedright\hspace{0pt}\arraybackslash}p{#1}}

\begin{center}\vspace*{-1em}\small
  \begin{tabular}{P{0.95in}P{2.825in}l}
    \toprule
    \textbf{Category} & \textbf{Name} & \textbf{COMBINE page or other reference}\\
    \midrule
    COMBINE representation standards
    & BioPAX (\emph{Biological Pathways Exchange})	& \URL{biopax}\\
    \\[-31pt]
    & CellML						& \URL{cellml}\\
    \\[-8pt]
    & SBGN (\emph{Systems Biology Graphical Notation})	& \URL{sbgn}\\
    \\[-8pt]
    & SBML (\emph{Systems Biology Markup Language})	& \URL{sbml}\\
    \\[-8pt]
    & SBOL (\emph{Synthetic Biology Open Language})	& \URL{sbol}\\
    \\[-8pt]
    & SED-ML (\emph{Simulation Experiment Description Markup Language}) &  \URL{sed-ml}\\
    \\[-10pt]
    \midrule
    \\[-10pt]
    Associated standardization efforts
    & BioModels.net Qualifiers				& \URL{qualifiers}\\
    \\[-31pt]
    & COMBINE Archive					& \URL{omex}\\
    \\[-8pt]
    & MIASE (\emph{Minimum Information About a Simulation Experiment}) & \URL{miase}\\
    \\[-8pt]
    & MIRIAM (\emph{Minimal Information Required In the Annotation of Models}) & \URL{miriam}\\
    \\[-8pt]
    & KiSAO (\emph{Kinetic Simulation Algorithm Ontology}) & \URL{kisao}\\
    \\[-10pt]
    \midrule
    \\[-10pt]
    Related standardization efforts
    & BioSharing					& \cite{sansone2012toward}\\
    \\[-31pt]
    & CNO (\emph{Computational Neuroscience Ontology}) 	& \cite{lefranc_2012}\\
    \\[-8pt]    
    & FieldML (\emph{Field Markup Language})		& \cite{christie_2009}\\
    \\[-8pt]
    & GPML (\emph{GenMAPP Pathway Markup Language})	& \cite{gpml_2014}\\
    \\[-8pt]    
    & MAMO (\emph{Mathematical Modeling Ontology})	& \cite{mamo_2014}\\
    \\[-8pt]    
    & NeuroML 						& \cite{gleeson_2010}\\
    \\[-8pt]    
    & NuML (\emph{Numerical Markup Language})		& \cite{dada_2010}\\
    \\[-8pt]    
    & PSI-MI (\emph{Proteomics Standards Initiative})	& \cite{hermjakob_2004}\\
    \\[-8pt]    
    & SpineML (\emph{Spiking Neural Markup Language})	& \cite{richmond2014model}\\
    \\[-8pt]    
    & TEDDY (\emph{TErminology for the Description of DYnamics}) & \cite{courtot2011a}\\
    \bottomrule
  \end{tabular}
\end{center}\vspace{-0.85em}

The differences in the categories are as follows:

\begin{itemize}\vspace*{-1em}

\item The \emph{COMBINE representation standards} meet a number of basic criteria which include the following: (i) represent information in biology, (ii) possess democratically-elected editorial boards, (iii) possess full specifications of version 1.0 or higher, (iv) have API library implementations supporting the standard, and (v) have continued development supported by a unified group of identifiable people.

\item The \emph{Associated standardization efforts} are either in a more fledgling state of development, or are not standard formats per se but rather tools or services that facilitate the use or interoperability of the COMBINE representation standards.

\item The \emph{Related standardization efforts} are other efforts that are either candidate COMBINE standards in early stages of development, or else are mature efforts in their own right that have their own substantial communities and, while not part of COMBINE, are efforts that the COMBINE community is involved in.

\end{itemize}\vspace*{-1em}

A standardization community that wants to become part of COMBINE begins as a \emph{Related standardization effort}.  The developers of the standard should join the COMBINE announcement mailing lists and, especially, participate in the annual COMBINE meetings discussed below.  If the effort meets the basic criteria to be considered a full-fledged standard, the \emph{COMBINE Coordinators} will accept the effort as a COMBINE standard.


\divider{The operation of COMBINE}

COMBINE, as an organization, currently performs the following activities:
\vspace*{-1em}
\begin{itemize}

\item \emph{Organize meetings}: COMBINE organizes open meetings where interested people can gather for face-to-face discussions and work on the standards.  The primary meetings are the annual COMBINE Forum and the annual HARMONY (HAckathon on Resources for MOdeliNg in biologY) workshop, held approximately six months apart.  The joint meetings help the different standardization efforts work together; they also make financial sense by reducing the overall number of meetings, travel, and money spent on hosting meetings.  (However, COMBINE does not currently have any funding of its own, and the meetings must be organized by groups that volunteer to host them.)  The leaders of the various standards also endeavor to write meeting reports that summarize the outcomes of the meetings~\citep[e.g.,][]{le2011meeting, waltemath2014meeting}.

\item \emph{Help coordinate standards development}: Thanks in large part to the meetings that COMBINE organizes, the discussion forums it provides, and the involvement of many of the same people in multiple standardization efforts, COMBINE helps coordinate the activities of the different efforts.  This reduces duplication of effort, user confusion, and non-interoperability among the efforts.

\item \emph{Identify missing standards and initiate efforts to develop them}: COMBINE's meta-community is in an ideal position to identify what is missing from the current constellation of standards in computational systems biology.  This has already yielded benefits: we have recently developed the COMBINE archive, a format that fills the need for a simple, consistent way of bundling multiple files related to a modeling project~\cite[\url{http://co.mbine.org/standards/omex};][]{Bergmann2014combine}; and we have also begun to identify missing minimal requirements for common annotations across the spectrum of data used in biological modeling, such as parameter identifiability (tentatively called the Minimal Information for Model Inference and Parametrisation---MIMIP) and mathematical classification (the Mathematical Modelling Ontology---MAMO).

\item \emph{Provide a specification infrastructure}: COMBINE provides a consistent framework for cataloguing the definitions of COMBINE standards. This framework includes a consistent, hierarchical identifier scheme for identifying standard specifications; a URI scheme for locating specifications and standards~\cite[using Identifiers.org to provide permanent, resolvable URIs for standards;][]{Juty2012}; and a web page structure for the description of each standard.

\item \emph{Develop common procedures}: Many standardization efforts are started by academics who have little experience with community organization.  Effective organization is something that takes time and experience to learn.  In COMBINE, we are documenting our experiences and collecting them into a collection of examples, recommendations and best practices (e.g., \url{http://co.mbine.org/Documents/criteria}).  We hope to provide would-be standards developers with a set of off-the-shelf ``standard operating procedures'' for different situations and goals.

\item \emph{Organize tutorials}: Educating biologists about available standards and compatible software tools is another important activity pursued by COMBINE.  We organize tutorials at the primary COMBINE meetings as well as at international conferences, in particular the annual \emph{International Conference on Systems Biology} (ICSB).

\item \emph{Maintain collective online forums/groups}:  COMBINE maintains mailing lists and online discussion forums (\url{http://co.mbine.org/comm}).  A discussion list cover the topic of general interest for all COMBINE members, while dedicated lists cover specific issues such as the COMBINE archive, metadata, etc.  General announces are done via social media (e.g., Twitter feed \emph{@combine\_coord}).

\end{itemize}

An additional activity that we hope to undertake soon is fund-raising.  This will require COMBINE to become a legal entity that can accept funding.  Once this is in place, we hope to be able to fund the meetings and online infrastructure, and perhaps also seek funding for further standards development.


\clearpage
\divider{Lessons learned in community-based standards development}

Taking a proposed standard from inception to adoption is not straightforward.  It is often difficult for academic researchers to organize successful community-oriented standards; the necessary skill development is not part of any science training curriculum, and there is a dearth of practical resources for standards development.  Moreover, the work of standards development does not end at adoption: successful standards must continue to evolve with changes in technologies and user needs.  (This has implications for funding, which we discuss further below.)

In what follows, we relate some of the lessons learned from our experiences developing community-based standards for biology.  Not all of these will generalize to other circumstances, but we hope that many will.


\subdivider{Ingredients for successful development of community-based standards}

% Production of an information or data communication standard is sometimes undertaken as a strategic move, particularly when commercial interests are involved.  Often, however, the desire for a standard arises when individuals and groups who want to collaborate discover they need a shared language, protocol, framework or other common structures to make it possible.

Different activities, different strategies and sometimes different personnel are needed at different stages in a standard's life.  The following are some of the vital ingredients shared by many of the most successful efforts:

\vspace*{-1em}
\begin{itemize}

\item \emph{Focus on significant, common problems}.  Useful standards are those that address a significant need face by many people, not just one laboratory or organization.  Sometimes a standard is a side-effect of other work.  For example, SBML was originally a by-product of an effort to develop a software interoperability framework: the framework needed a data exchange format.  It was not until later that it became clear that SBML addressed a real need for many people, and that it should be the focus of a standardization effort in its own right.  Similar stories underlie behind many of the COMBINE standards.

\item \emph{Start development with a subset of possible stakeholders}.  The early stages of a standard's development are more efficient if undertaken by a small number of stakeholders.  The developers need to possess strong technical skills, creativity in solving problems, and enough experience to avoid falling into common traps (\eg overengineered solutions, reliance on sexy but untested technologies, etc.).  Design by a small group of domain experts stands a better chance of succeeding in laying a coherent foundation.  This admittedly runs against the goal of eventually having a democratic, open process, but it is a temporary bootstrapping phase.

\item \emph{Open up the process once the standard gains traction}.  Once adoption increases, it is no longer in the best interests of the user community to have the standard be controlled by a group of self-appointed authorities.  This is the time to introduce a democratic process, with an editorial board elected from the community, a system for proposing changes to the standard, and other community-oriented processes.  This helps spread the intellectual work load in continuing to evolve the standard, and it lets the community take ownership, which encourages increased involvement and adoption.

\item \emph{Use a staged development approach}.  Attempting to support the kitchen sink in a standard right from the start is a recipe for failure, but so is producing an incomplete standard.  An alternative is to borrow a tactic from product development and create the equivalent of a ``minimal viable product'' at each stage.  A simple standard that is fully usable for a subset of possible use-cases will allow people to start using it in practice, which in turn produces feedback for the next stage and simultaneous begins creating a user base.

\item \emph{Engage the community repeatedly}.  Workshops are an essential vehicle for getting more involvement by the community.  They provide an opportunity to expose other people to the standard as well as get ideas for improvements.  They also allow standards developers to demonstrate how the standards are meant to be used (which is not always as clear to readers of specification documents as the authors may think!).

\item \emph{Make sure there is a shepherd}. Volunteer contributions are essential for a community-oriented process, but long-running projects often need more.  Many standardization projects are successful due in part to having someone who devotes a majority of their time to the effort as a whole: mediating disputes in the community, leading a development team, seeking funding, etc.  The involvement of the same person(s) over many years also provides crucial continuity to the effort.  Unfortunately for those in that role, that time is spent on activities that do not produce high-profile research publications---the primary metric for academic success.

\end{itemize}
\vspace*{-1em}

It may be surprising that most of the ingredients above are actually about human-level concerns than technological concerns.  This reflects the fact that standards development is at least as much about reaching agreement between people as it is about the technical details of the standard.


\clearpage
\subdivider{Community organization structures for small-scale efforts}

Continued maintenance and evolution of a community standard requires some kind of governance structure.  There are many general standards organizations today, ranging from large and well-known entities such as W3C to more specialized entities in different sectors, that could provide governance structure, operating processes, and other organizational elements.  However, these generally require more funding for participation and staffing than smaller academic standards efforts can muster.

Many of the efforts in COMBINE invented their own governance structure and community development processes precisely because getting involved with larger organizations was beyond their resources.  The SBML process is probably the most evolved from among those in COMBINE, and is detailed online at SBML.org~\cite{sbml-process}.  The process was originally based on W3C's, but drastically simplified and reduced to make it more appropriate to the scale of the community.  Here are some points in the process:

\vspace*{-1em}
\begin{itemize}

\item The community is organized into the \emph{SBML Forum}, the \emph{SBML Editors}, and the \emph{SBML Team}. The SBML Forum consists of all members of the community who subscribe to the sbml-discuss mailing list, with the list membership acting as a kind of basic voter registration mechanism.  The SBML Editors are volunteers who are sufficiently interested in SBML that they are willing to spend time in the development, writing, and correction of SBML specification documents.  There are five SBML Editors at any given time; they are elected by a majority vote from among the SBML Forum, and they serve three year terms, with reelection being possible but consecutive terms being disallowed.  The SBML Team are members who are employed to work on SBML-related activities.  Their tasks include maintaining the resources that support the SBML community and SBML development in general, developing critical software, and other similar activities.
 
\item Discussions about SBML are held publicly as much as possible, usually on the sbml-discuss mailing list and in face-to-face meetings.  The public discussions and archives improve transparency, provide a public record of arguments and reasoning, and stimulate the broader community.  Similarly, in technical decisions, consensus is sought as much as possible. In situations where a decision appears to have no obvious right or wrong answer on technical grounds alone, the SBML Editors may initiate a public vote on the matter.

\item A process document details the procedures used for continued development of SBML, election of editors, voting on issues, and much more.  This helps the process be objective.

\end{itemize}
\vspace*{-1em}

The main ingredients of this process have been adopted by other COMBINE standardization efforts.  In COMBINE, our ambition is to develop a set of \emph{standard operating procedures} that encapsulate the procedures and guidelines that any standardization effort can use off-the-shelf to start their effort.


\divider{Opportunities for NIH}


\subdivider{Funding for development of standards}

\subdivider{Large-scale coordination}

\subdivider{Long-term maintenance}
